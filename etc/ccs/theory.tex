\documentclass{article}
\usepackage[utf8]{inputenc}

\title{Compression security theoretical underpinnings}
\author{
Dimitris Karakostas
\and
Aggelos Kiayias
\and
Dionysis Zindros
}
\date{April 2016}

\usepackage{natbib}
\usepackage{amsmath}
\usepackage{graphicx}
\usepackage{listings}
\usepackage{mathtools}
\usepackage{amsfonts}
\usepackage{amsthm}
\usepackage{amssymb}

\DeclarePairedDelimiter{\ceil}{\lceil}{\rceil}
\let\oldemptyset\emptyset
\let\emptyset\varnothing

\newtheorem*{lemma}{Lemma}

\newcommand\defeq{\stackrel{\mathclap{\normalfont\mbox{def}}}{=}}

\begin{document}

\maketitle

\section{Reflection security}

\subsection{Adaptive reflection game}

Let $\mathcal{PE} = (\mathcal{K}, \mathcal{E}, \mathcal{D})$ be a public-key
encryption scheme, $\mathcal{A}$ be an adversary and $\mathcal{S}$ be a
simulator of $\mathcal{A}$.  The game
$\text{Game}_{\text{REF-SEC}}^{\mathcal{PE},\mathcal{A}}(\lambda,  f,
\mathcal{V}, g)$ is parameterized with a rendering function $f(\cdot, \cdot,
\cdot)$, a noise distribution $\mathcal{V}$, the security parameter $\lambda$,
some function $g$ of the plaintext, and a distribution of secrets $\mathcal{M}$
such that $\forall s \in \mathcal{M}: |s| = \lambda$.

At the initialization phase of the game, the challenger produces a
$\lambda$-bit key $(pk, sk) \leftarrow \mathcal{K}(\lambda)$. The adversary is
given $pk$, $\mathcal{V}$, $g$ and $f$.  The challenger then chooses a secret
$s \leftarrow \mathcal{M}$.

After the initialization phase of the game, the game is played in rounds. At
the beginning of each round $i$, the adversary chooses a reflection string
$r_i$ and sends it to the challenger. The challenger produces a noise string
$v_i \leftarrow \mathcal{V}$ and computes $m_i = f(s, r_i, v_i)$.  Subsequently
$m_i$ is encrypted as $c_i = \mathcal{E}_\kappa(m_i)$, and $c_i$ is sent back
to the adversary. At the end of the round, the adversary  decides if more
rounds are needed, and if so, more rounds are played at the discretion of the
adversary.

When the adversary decides to complete the game, they output a guess $y$. The
adversary is successful if $g(s) = y$.

Formally, let the public key adversarial game be defined as follows:

\begin{lstlisting}[texcl,mathescape]
def $\text{Game}_{\text{REF-SEC}}^{\mathcal{PE},\mathcal{A}}(\lambda, f, \mathcal{V}, \mathcal{M}, g)$:
    $(pk, sk) \leftarrow \mathcal{K}(\lambda)$
    $st \leftarrow \mathcal{A}(\text{"init"}, pk, f, \mathcal{V}, \mathcal{M}, g)$
    $s \leftarrow \mathcal{M}$
    $y = \bot$
    while $y = \bot$:
        $(st, r) \leftarrow \mathcal{A}(\text{"reflect"}, st)$
        $v \leftarrow V$
        $m = f(s, r, v)$
        $c = \mathcal{E}_{pk}(m)$
        $(st, y) \leftarrow \mathcal{A}(\text{"play"}, st, c)$
    if $y = g(s)$:
        return 1
    else:
        return 0
\end{lstlisting}

Let the simulator game be defined as follows:

\begin{lstlisting}[texcl,mathescape]
def $\text{Game}_{\text{REF-SIM}}^{\mathcal{PE},\mathcal{S}}(\lambda, f, \mathcal{V}, \mathcal{M}, g)$:
    $y \leftarrow \mathcal{S}(f, \mathcal{V}, \mathcal{M}, g)$
    $s \leftarrow \mathcal{M}$
    if $g(y) = s$:
        return 1
    else:
        return 0
\end{lstlisting}

Let

\begin{equation*}
    \text{Adv}_{\mathcal{PE}, \mathcal{A}, \mathcal{S}}(\lambda, f, \mathcal{V}, \mathcal{M}, g)
    \defeq
    |Pr[\text{Game}_{\text{REF-SEC}}^{\mathcal{PE},\mathcal{A}}(\lambda, f, \mathcal{V}, \mathcal{M}, g) = 1]
    -
    Pr[\text{Game}_{\text{REF-SIM}}^{\mathcal{PE},\mathcal{S}}(\lambda, f, \mathcal{V}, \mathcal{M}, g) = 1]|.
\end{equation*}

The above game can be modified to deal with symmetric encryption. For this,
the choice of key is changed to be limited to a secret key instead of a public and private key
pair. The adversary is not given access to any key material. Instead,
the adversary is given access to an encryption oracle which encrypts a given message
using the secret key and returns the respective ciphertext. This oracle can be
called an arbitrary number of times by the challenger.

In the following proofs, we will assume that the game
played is symmetrically. As the attack does not pertain to the key exchange mechanism,
all theorems can be directly translated between the two versions of the game.

\subsection{Adaptive reflection security}

Given a polynomially computable rendering function $f(\cdot, \cdot, \cdot)$ and
a  noise distribution $\mathcal{V}$, a public-key encryption scheme
$\mathcal{PE}$ is \textit{reflection-secure} if:

\begin{equation*}
\forall \mathcal{M}:
\forall g:
\forall PPT \mathcal{A}:
\exists PPT \mathcal{S}:
\text{Adv}_{\mathcal{PE}, \mathcal{A}, \mathcal{S}}(\lambda, f, \mathcal{V}, \mathcal{M}, g) = negl(\lambda)
\end{equation*}

\section{The security of compression}

\subsection{Compression functions}

A compression function $Com$ is a polynomial polynomially reversible $1 - 1$
function that takes an input $x$ and outputs $y$. $x$ is a random variable that
follows the distribution $X$. $y = Com(x)$ will follow the
distribution $Com(X)$.  $E(|y|) / E(|x|) < 1$ is the metric that defines the
function's performance, where $E$ denotes expectation.

\subsection{Composing with compression}

We now shift our interest to algorithms which compose compression and
encryption.  Therefore, we treat the public-key encryption scheme above as a
composition of a compression and encryption algorithm:

\begin{equation*}
    \mathcal{E} = Enc \circ Com
\end{equation*}

$Com$ denotes a compression function as defined above, while $Enc$ denotes an
encryption function. We stress our important assumption that $Enc$ is defined
on $\{0, 1\}^*$ and not on some fixed-length input.

\subsection{Length monotonicity}

An important assumption for our attack is that the encryption function $Enc$ is
\textit{length monotonic}. Specifically,

\begin{equation*}
\begin{split}
\forall m_1, m_2 \in \{0, 1\}^*\\
\forall \kappa_1, \kappa_2 \in \mathcal{K}\\
|m_2| > |m_1|
\Rightarrow
|Enc_{\kappa_1}(m_2)| \geq |Enc_{\kappa_2}(m_1)|
\end{split}
\end{equation*}

If the last inequality becomes strict, we call the encryption function
\textit{strictly length monotonic}.

A practical instance of a length monotonic encryption function is AES.
Stream ciphers can be treated as strictly monotonic to the bit level.

\section{Property compressibility}

We now define what it means for a property $Q$ of the plaintext $x$ chosen out of the distribution $\mathcal{M}$ to be \textit{compression-detectable} by plaintext $y$ under a compression function $Com$ rendered under some rendering function $f$ together with some noise chosen out of the noise distribution $\mathcal{V}$:

\begin{equation*}
    \begin{split}
        \exists y \in Y\\
        Pr[
            |Com(f(x_1, y, v_1))| < |Com(f(x_2, y, v_2))|
        ]
        \geq
        \frac{1}{2} + \alpha(\lambda)
    \end{split}
\end{equation*}

Where $\alpha(\lambda)$ is a non-negligible function and
the probability is taken over the following random variables:

\begin{align*}
    x_1 &\leftarrow \mathcal{M}|_Q\\
    x_2 &\leftarrow \mathcal{M}|_{\lnot Q}\\
    v_1 &\leftarrow \mathcal{V}\\
    v_2 &\leftarrow \mathcal{V}
\end{align*}

Where $\mathcal{M}|_Q$ denotes the subset of $\mathcal{M}$ restricted
over the elements for which $Q$ holds proportionally retaining measure.

Furthermore we require that such a $y$ is polynomially computable given
the property $Q$, the functions $Com$ and $f$, the set $Y$ and the distributions
$\mathcal{M}$ and $\mathcal{V}$ by a function denoted $\mathcal{O}_R(Com, f, Q, \mathcal{M}, \mathcal{V})$.

Note that compression detectability implies that the distributions of lengths,
$|Com(f(x_1, y, v_1))|$ and $|Com(f(x_2, y, v_2))|$, have non-negligible statistical
distance with an efficiently computable distinguisher.

The compression-detectibility property exists in all common compression function
as long as $x$ and $y$ compress in the same context. We will illustrate such examples later in this work.

\section{The BREACH model}
We now turn our attention to the BREACH attack specifically, which utilizes
the properties explored above, and we show that, under ideal modeling,
the above attack is instantiated when the TLS algorithms are in use.

In the BREACH model, we deal with a specific function $f$, property $Q$,
compression function $Com$, length-leaking $Enc$, and secret distribution
$\mathcal{M}$.

In the next paragraphs, we describe a simplified model of the attack that
lends itself to formal theoretical proofs, while abstracting away the complex
practical details of the attack pertaining to the networking and particularities
of the implementation that are of little significance to the cryptographic
formalisms presented here.

\subsection{The rendering context assumption}

In the modern web, function $f$ represents a specific attack target, which
constitutes the renderer of a web page, given a specific secret, reflection,
and noise. Typically, $f$ will concatenate instances of the secret, the
reflection, and the noise, between some context strings which constitute the
static portions of the site.

We assume that $f$ outputs a concatenation of various instances of
$s$, $r$, and $v$ with arbitrary repetitions, in addition to concatenating with some
constant context whose position is independent of the values of $s, r, v$.
The number of repetitions of $s, r, v$ are also independent of the values of
$s, r, v$.

Formally, assume that the symbols $\{\beta_1, \beta_2, \beta_3\}$ do not appear in the
alphabet of the secret.

Our assumptions for $f$ can be stated as follows. Let

\begin{equation*}
\begin{split}
\exists \mu \in \{0, 1, \beta_1, \beta_2, \beta_3\}^*\\
\exists u_1, v_1, u_2, v_2, u_3, v_3 \in \{0, 1, \beta_1, \beta_2, \beta_3\}^*\\
\mu = u_1 \beta_1 v_1 \land\\
\mu = u_2 \beta_2 v_2 \land\\
\mu = u_3 \beta_3 v_3 \land\\
\forall s, r, v \in \{0, 1\}^*\\
f(s, r, v) = Rep(Rep(Rep(\mu, \beta_1, s), \beta_2, r), \beta_3, v)
\end{split}
\end{equation*}

Where $Rep$ is the string replacement function:

\begin{align*}
Rep(\epsilon, \alpha, \beta) &\defeq \epsilon\\
Rep(\alpha:xs, \alpha, \beta) &\defeq \beta:Rep(xs, \alpha, \beta)\\
Rep(x:xs, \alpha, \beta) &\defeq x:Rep(xs, \alpha, \beta)
\end{align*}

Where $:$ denotes the concatenation of a symbol with a string.

We call $f$ functions that exhibit such behavior \textit{concatenative}.

\subsection{The compression function assumption}

In BREACH, $Com$ is a variant of LZ77. In this compression algorithm, repetitions
of strings are well-compressed. In our ideal model, we examine a simplified version
of LZ77 which we describe in this paragraph.

Our version of LZ77 is not composed with a symbol-compressing algorithm such as
Huffman coding. Therefore, a way is needed to encode literal symbols and compression
symbols. The LZ77 algorithm treats the uncompressed text as a series of bits. Let the
plaintext be $p$ and its length be $n = |p|$. The LZ77 compression function needs
to have an alphabet of 2 symbols for representing literals (the bits $0$ and $1$ of
the plaintext) as well as an alphabet of an additional $n$ symbols for representing compression distances from $1$ to $n$ and lengths from $1$ to $n$ (the same symbol is
used to represent a distance and a length). This makes an alphabet of a total of
$n + 2$ symbols which are represented by bit sequences of exactly $\ceil{lg(n + 2)}$
bits each.

We first describe the decompression algorithm. At the beginning of the compressed
string, the number
of characters in the uncompressed plaintext is encoded in ASCII followed by the "$\$$"
ASCII character as a separator. This allows the decoder to compute how many bits are used to
represent each symbol in the compressed text.

Ordering these bit strings lexicographically, the first two bit strings will correspond to the literal
"$0$" and "$1$" bits of the plaintext. The next bit strings will correspond to \textit{distance} bit sequences of
values $1, 2, \ldots, n$ respectively.

After the "$\$$" separator, an encoding of the plaintext is given. The encoding
can be decompressed by linearly reading it one character at a time. If a bit sequence
corresponding to a literal
is given, this is decoded as the respective literal bit of the plaintext. If
a distance bit sequence of value $a$ is given, then another distance bit sequence of value $b$ sequence follows.
Then $a$ must be less than or equal to the plaintext characters decoded so far, while $b$
is arbitrary. This pair of values is decoded as follows. The decompressed text beginning at a distance of $a$ bits behind the current position decoding position and of length $b$ bits is taken into account. That text is then the new decompressed plaintext which is appended to the existing decompressed plaintext one bit at a time. This
procedure can cause self-overlapping texts. In the end, the decoded text must be $n$
bits long, otherwise the compression is invalid.

\begin{lstlisting}[texcl,mathescape]
def $\text{LZ77}^{-1}(c)$:
    parts = c.split("$\$$")
    plaintextLength = parse_int(parts[0])
    bitsPerSymbol = $\ceil{lg(\text{plaintextLength}) + 2}$
    encodedPlaintext = stringToBinaryArray("".join(parts[1:]))

    plaintext = ''
    for i in range(0, len(encodedPlaintext), bitsPerSymbol):
        symbol = binaryToInt(encodedPlaintext[i:bitsPerSymbol])
        if symbol = 0:
            plaintext += '0'
        else if symbol = 1:
            plaintext += '1'
        else:
            distance = symbol - 1
            i += bitsPerSymbol
            length = int(encodedPlaintext[i:bitsPerSymbol])
            if distance > len(plaintext):
                return $\bot$
            while length:
                plaintext += plaintext[-distance]
                length -= 1
    if len(plaintext) != plaintextLength:
        return $\bot$
    else:
        return plaintext
\end{lstlisting}

Where negative and range array indexing and the $range$ function is as in Python.

Our LZ77 compressor can then be described simply: It chooses the shortest
in bits compressed text, which, when decoded
using the above algorithm yields the desired plaintext.

% TODO: Prove that it is a compression function for some $\mathcal{M}$

\subsection{The full-text predicate assumption}

The BREACH attack is a specific instance of reflection-based attacks in which
the $Q$ property pertains to a prefix of the secret. It asks "Does the secret
begin with such text?"

We require $X$ to be of compressible length,
i.e. the following equation must be satisfied:

\begin{equation*}
    |X| > \ceil{lg(f(X, X, \epsilon) + 2)}
\end{equation*}

In practical terms, this requires the secret that needs to be stolen
to be at least 3 bytes so that it is better compressed rather than
represented as a literal.

\subsection{BREACH prefixes are compression-detectable}

Let $startswith(v, s)$ be the $k$-length prefix checking predicate on a $\lambda$-bit
secret $s$:

\begin{equation*}
    startswith(v, s) \defeq \exists w: s = v w
\end{equation*}

\begin{lemma}

Let $f$ be a concatenative rendering function and let $Com = LZ77$. Then:

\begin{equation}
Pr[startswith(v, \cdot) \textrm{ compression-detectable under } f, Com] = \text{non-negl}(\lambda - k)
\end{equation}

Where the probability randomness is taken uniformly over all $k$-bit prefixes $v \in \{0, 1\}^k$.

\end{lemma}

\begin{proof}
?
\end{proof}

\section{The compression distinguishing attack}

\begin{lemma}

Let $Com$ be a compression function, $f$ be a rendering function and $Q$
be a plaintext predicate. If $Q$ is compression-detectable under a rendering
of $f$ compressed with $Com$
under plaintext distribution
$\mathcal{M}$, reflection alphabet $\mathcal{M}$ and noise distribution
$\mathcal{V}$, then:

\begin{align*}
\forall Enc:
\exists g:
\exists PPT \mathcal{A}:
\forall PPT \mathcal{S}:
\text{Adv}_{\mathcal{SE}(Enc, Com), \mathcal{A}, \mathcal{S}}(\lambda, f, \mathcal{V}, \mathcal{M}, g) = P(\lambda)
\end{align*}

Where $P$ is a non-negligible function and $Enc$
is strictly length-monotonic.

\end{lemma}

\begin{proof}

Let $g$ be the boolean function $Q$ on the plaintext. Define the adversary
$\mathcal{A}$ as follows:

\begin{lstlisting}[texcl,mathescape]
def $\mathcal{A}("init")$:
    return $\bot$

def $\mathcal{A}("reflect", Com, f, Q, \mathcal{M}, \mathcal{V})$:
    $r \leftarrow \mathcal{O}_R(Com, f, Q, \mathcal{M}, \mathcal{V})$
    return $r$

def $\mathcal{A}("play", Q, \mathcal{M}, c, y)$:
    $s_1 \leftarrow \mathcal{M}|_Q$
    $s_2 \leftarrow \mathcal{M}|_{\lnot Q}$

    $v_1 \leftarrow \mathcal{V}$
    $v_2 \leftarrow \mathcal{V}$

    $l_1 = |Enc_\epsilon(Com(f(s_1, r, v_1)))|$
    $l_2 = |Enc_\epsilon(Com(f(s_2, r, v_2)))|$

    if $|c| < l_1$:
        return True
    else if $|c| > l_2$:
        return False
    else:
        return $\bot$
\end{lstlisting}

Then let $\mathcal{S}$ be an arbitrary simulator. Since the probability of simulator
success is bounded by $?$, since its output is independent of the choice of
$s$.

\begin{equation}
    \forall \mathcal{S}:
    Pr[\text{Game}_{\text{REF-SIM}}^{\mathcal{PE},\mathcal{S}}(\lambda, f, \mathcal{V}, \mathcal{M}, g) = 1] \leq 0.5
\end{equation}

But the probability of success of $\mathcal{A}$ is $\frac{1}{2} + \alpha(\lambda)$ for some non-negligible function $\alpha$ (proof?).

Therefore,

\begin{equation}
\begin{split}
    \forall PPT \mathcal{S}:
    \text{Adv}_{\mathcal{SE}(Enc, Com), \mathcal{A}, \mathcal{S}}(\lambda, f, \mathcal{V}, \mathcal{M}, g) \geq ?
\end{split}
\end{equation}

Which is non-negligible. This completes the proof.

\end{proof}

We can now amplify the attack to achieve a better probability of success by a small modification in our adversary. For readability, we make use of a global adversary state $st$. Formally, this global state would be carried between adversary invocations as a function argument. The amplification is parameterized by a parameter $k$.

\begin{lstlisting}[texcl,mathescape]
global $st$

def $\mathcal{A}("init")$:
    $st = \{i: 0, low: 0, high: 0\}$
    return $\bot$

def $\mathcal{A}("reflect", Com, f, Q, \mathcal{M}, \mathcal{V})$:
    $r \leftarrow \mathcal{O}_R(Com, f, Q, \mathcal{M}, \mathcal{V})$
    return $r$

def $\mathcal{A}("play", Q, \mathcal{M}, c, y)$:
    $\{i, low, high\} = st$

    $s_1 \leftarrow \mathcal{M}|_Q$
    $s_2 \leftarrow \mathcal{M}|_{\lnot Q}$

    $v_1 \leftarrow \mathcal{V}$
    $v_2 \leftarrow \mathcal{V}$

    $l_1 = |Enc_\epsilon(Com(f(s_1, r, v_1)))|$
    $l_2 = |Enc_\epsilon(Com(f(s_2, r, v_2)))|$

    if $|c| < l_1$:
        $low = low + 1$
    else if $|c| > l_2$:
        $high = high + 1$

    $i = i + 1$
    if $i > k$:
        if $low > high$:
            return True
        else if $high > low$:
            return False
        else:
            return $\bot$
    else:
        $st = \{i, low, high\}$
        return $\bot$
\end{lstlisting}

Proof that this amplifies?

\subsection{Compression is insecure}

We now show that all reasonable compression function expose a compression-detectable property $Q$. Indeed, if a compression function does not expose any compression-detectable properties $Q$, then the compression function does not cross-compress the data in $X$ with the data in $Y$ (Why?).

\section{Mitigation of compression distinguishing attacks}

\subsection{Mitigation against BREACH}

\subsubsection{Disabling compression}
Compression is the basic assumption for property compressibility. Disabling
compression would break the adaptive reflection game, since it removes
the compression part of the encryption function.

\subsubsection{SameSite cookies}
SameSite cookies \cite{c1} is a proposed draft for enhancing web security. This policy
states that cookies may only be included in same origin requests. Enabling
SameSite cookies would result in the elimination of cross-origin requests as a
method of implementing the "play" oracle of the Reflection-Security Game.

\subsubsection{Compress secrets in distinct contexts}

\subsubsection{Masking secrets}

\subsubsection{Length hiding}

\subsubsection{Rate limiting/monitoring}

\begin{thebibliography}{1}

\bibitem{c1} M. West, First-Party-Only Cookies, RFC Internet-Draft, 2015.

\end{thebibliography}

\end{document}
